%\documentclass[a4paper,12pt]{article}
%\documentclass[platex,dvipdfmx]{jlreq}			% for platex
\documentclass[uplatex,dvipdfmx]{jlreq}		% for uplatex
\usepackage{amsmath,amssymb,amsfonts}
\usepackage{bm}
\usepackage{array}
\usepackage{lmodern}
\usepackage{hyperref}
\usepackage{comment}
\usepackage{xcolor} % 色指定
\usepackage{tcolorbox} % 問題の枠組み用

% ページ設定
\setlength{\textwidth}{450pt}
\setlength{\oddsidemargin}{0pt}
\setlength{\evensidemargin}{0pt}
\setlength{\textheight}{650pt}
\setlength{\topmargin}{0pt}

\title{Linear Algebra Exercises Part01}
\author{Taka.N}
\date{\today}

% 問題用のスタイル定義
\newtcolorbox{problem}[1][]{
  colframe=gray!80,       % 外枠: ダークグレー
  colback=gray!20,        % 背景: ライトグレー
  coltitle=black,         % タイトル: 黒
  boxrule=0.8pt,         
  sharp corners,
  title=#1
}

% 解答スタイルの定義
\newenvironment{solution}{
  \par\noindent\textbf{Solution:} \itshape
}{\vspace{10pt}}

\begin{document}

\maketitle

% 目次を追加
\tableofcontents
\newpage % 目次の後に改ページ

\section{Problem 1.1}
\begin{problem}[Prob1.1]
行列$X, Y$が同サイズで積$XY$が定義できるとする。  
$(XY)^T = Y^T X^T$であることを示せ。
\end{problem}

\begin{solution}  
$XY$の$(i,j)$成分を$(XY)_{ij}$とする。このとき、  
\[
(XY)_{ij} = \sum_{k} X_{ik} Y_{kj}.
\]  
転置をとると、  
\[
(XY)^T_{ij} = (XY)_{ji} = \sum_k X_{jk} Y_{ki}.
\]
一方で$Y^T X^T$の$(i,j)$成分は  
\[
(Y^T X^T)_{ij} = \sum_k (Y^T)_{ik} (X^T)_{kj} = \sum_k Y_{ki} X_{jk}.
\]
よって$(XY)^T = Y^T X^T$が示された。
\end{solution}

\section{Problem 1.2}
\begin{problem}[Prob1.2]
行列$X, Y, Z$が適切なサイズで積$XYZ$が定義できるとする。  
$(XYZ)^T = Z^T Y^T X^T$であることを示せ。
\end{problem}

\begin{solution}
$(1)$の結果を二度適用することで示す。  
\[
(XYZ)^T = (X(YZ))^T = (YZ)^T X^T.
\]
さらに$(YZ)^T$に$(1)$を適用すると、  
\[
(YZ)^T = Z^T Y^T.
\]
したがって、  
\[
(XYZ)^T = Z^T Y^T X^T.
\]
\end{solution}

\section{Problem 1.3}
\begin{problem}[Prob1.3]
$n \times n$行列$A$が可逆であるとき、$A^T$も可逆であり、$(A^T)^{-1} = (A^{-1})^T$であることを示せ。
\end{problem}

\begin{solution}
$A$が可逆であるならば、$A^{-1}$が存在する。このとき、  
\[
(A^T)(A^{-1})^T = (A^{-1}A)^T = I^T = I.
\]
したがって、$(A^{-1})^T$は$A^T$の逆行列となる。よって、  
\[
(A^T)^{-1} = (A^{-1})^T.
\]
\end{solution}

\end{document}
